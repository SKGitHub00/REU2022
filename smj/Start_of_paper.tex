%%%
%%%  LaTeX template for publications
%%%  to be submitted to Statistical Modelling
%%%
%%%  Prepared by Arnost Komarek
%%%  Version 0.2 (20140214)
%%%    0.2:  style of references slightly changed,
%%%          support for use with bibTeX added
\documentclass[submit]{smj}


%%%%% PREAMBLE
%%%%% =============================================================================


%%% Place for putting personal \usepackage and \newcommand commands
%%% Note that some packages are loaded automatically
%%% with the smj class. 
%%% These include: graphicx, color, fancyvrb, amsmath, amssymb, calc, upquote (if available), natbib, url, hyperref.
%%%
%%% Please, specify all your personal definitions, newcommand etc. here
%%% and not inside the main body of the text.
%%% -------------------------------------------------------------------------------
%\usepackage{PACKAGE}
%\newcommand{MYCOMMAND}{...}


%%% Identification of authors
%%% -------------------------------------------------------------------------------
%%% For each author, provide his/her first name, surname and possibly initials 
%%% of the middle names. 
%%%
%%% Use \Affil{NUMBER} following the author name for each unique affiliation,
%%% where NUMBER is integer starting from 1 to the number of affiliations needed
%%% in this paper. In case of multiple affiliations of one author, use
%%% \Affil{NUMBER1,}\Affil{NUMBER2,}\Affil{NUMBER3} following the author's name
%%% as it is done for Emmanuel Lesaffre below.

  %%% For papers with 3 or more authors:
  %%%  in \Author{}, separate the authors with commas, the last author is separated by `and' without a comma,
  %%%  in \AuthorRunning{}, use the full name of the first author followed by \textrm{et al.}.
\Author{Kathryn Haglich\Affil{1}, Jeff Liebner\Affil{2},
        Sarah Neitzel\Affil{3}, 
        and Amy Pitts\Affil{4}
        %and Emmanuel Lesaffre\Affil{4,}\Affil{5}
}
\AuthorRunning{Haglich, Neitzel, Pitts}

  %%% For papers with 2 authors:
  %%%  in both \Author{} and \AuthorRunning{},
  %%%  use the full names of both authors separated by 'and' without a comma.
%\Author{Arno\v{s}t Kom\'arek\Affil{1} and Brian Marx\Affil{2}}
%\AuthorRunning{Arno\v{s}t Kom\'arek and Brian Marx}
  
  %%% For papers with 1 author:
  %%%  in both \Author{} and \AuthorRunning{},
  %%%  use the full name the author.
%\Author{Arno\v{s}t Kom\'arek\Affil{1}}
%\AuthorRunning{Arno\v{s}t Kom\'arek}


%%% Affiliations as they should appear on the title page.
%%% -------------------------------------------------------------------------------
%%% Do not provide the full addresses here.
%%% The ordering inside \Affiliations{} should correspond to NUMBERs used 
%%% in \Affil{} commands in \Author{}
\Affiliations{

  %%% 1
\item Department of Mathematics, 
      Lafayette College,
      Easton,
      Pennsylvania, USA

 %%% 2
\item Department of Mathematics, 
	  Faculty of Mathematics,
      Lafayette College,
      Easton,
      Pennsylvania, USA
  
  %%% 3    
\item School of Biodiversity Conservation,
      Unity College, 
      Unity,
      Maine, USA

  %%% 4
\item Department of Mathematics,
      Marist College,
      Poughkeepsie
      NY, USA

  
%\item Department of Biostatistics,
%      Erasmus University Rotterdam,
%      Rotterdam,
%      the Netherlands

  %%% 5
%\item Interuniversity Institute for Biostatistics and Statistical Bioinformatics,
%      KU Leuven and Universiteit Hasselt,
%      Leuven,
%      Belgium
}   %% end \Affiliations


%%% Postal, e-mail address, phone and fax of the corresponding author (not necessarily the first author).
%%% ------------------------------------------------------------------------------------------------------
%%% Use command \CorrAddress{} to provide a full postal address of the
%%% corresponding author in the form
%%% "Firstname Lastname, Department, University, Street 1, ZIP City, Country" 
%%% Use command \CorrEmail{} to provide an e-mail address of the corresponding author.
%%% Use command \CorrPhone{} to provide a phone number (including the country code!) of the corresponding author.
%%% Use command \CorrFax{} to provide a fax number (including the country code!) of the corresponding author.
\CorrAddress{Arno\v{s}t Kom\'arek, 
             Department of Mathematics, 
             Faculty of Mathematics,
             Lafayette College, 
             Easton,
             Pennsylvania, 
        		USA}
\CorrEmail{liebnerj@lafayette.edu}
\CorrPhone{(+420)\;221\;913\;282}
\CorrFax{(+420)\;222\;323\;316}


%%% Title and a short title (to be used as a running header) of the paper
%%% -------------------------------------------------------------------------------
\Title{Developing a Bayesian method for locating breakpoints in time series data.}
\TitleRunning{Template paper}


%%% Abstract
%%% -------------------------------------------------------------------------------
\Abstract{
An abstract of up to 200 words should precede the text together with 5 or 6 keywords in alphabetical order 
to describe the content of the paper. Authors should take great care in preparing the abstract and not simply 
lift it from the main text. The abstract should describe the background and contribution of the manuscript 
and give a~clear verbal description of the results and examples, and avoid citations whenever possible. 
Any acknowledgements will be printed at the end of the text.
}


%%% Key words
%%% -------------------------------------------------------------------------------
\Keywords{
keyword a; keyword b; keyword c; keyword d; keyword e
}


%%%%% MAIN BODY 
%%%%% =============================================================================
\begin{document}


%%% Title page
%%% -------------------------------------------------------------------------------
%%% Use command\maketitle to produce the title page.
\maketitle


%%% Main text
%%% ------------------------------------------
\section{Introduction}

\section{Math}
Here is the math for our paper

\subsection{Derivations of Ratio}


To start we need to find the ratio \\ 
\begin{align*}
ratio &= \frac{g(\tau_{n} K_{n} | x_1,\dots,x_n) }{g(\tau_{o} K_{o} | x_1,\dots,x_n)} \times \frac{q(\tau_{o} K_{o} | \tau_{n} K_{n})}{q(\tau_{n} K_{n}| \tau_{o} K_{o})} \\
&= \frac{\Big[ \frac{f(x_1,\dots,x_n | \tau_{n} K_{n}) \pi(\tau_{n} K_{n})}{\int f(x_1, \dots ,x_n | \tau_{n} K_{n}) \pi(\tau_{n} K_{n}) d \tau_{n} K_{n} } \Big] q(\tau_{o} K_{o} | \tau_{n} K_{n})}{\Big[ \frac{f(x_1,\dots,x_n | \tau_{o} K_{o}) \pi(\tau_{o} K_{o})}{\int f(x_1, \dots ,x_n | \tau_{o} K_{o}) \pi(\tau_{o} K_{o}) d \tau_{o} K_{o} }\Big] q(\tau_{n} K_{n}| \tau_{o} K_{o})} 
\end{align*}

Then we have,
\begin{align*}
ratio &= \frac{\Big[ \frac{ \int f(x_1,\dots,x_n | \tau_{n} K_{n}) \big(\pi(\tau|K)\pi(K)\pi(\beta)\pi(\sigma)\big)_{new} d\tau dK }{\int f(x_1, \dots ,x_n | \tau_{n} K_{n}) \big(\pi(\tau|K)\pi(K)\pi(\beta)\pi(\sigma)\big)_{new} d\tau dK d \beta d\sigma } \Big] q(\tau_{o} K_{o} | \tau_{n} K_{n})}{\Big[ \frac{ \int f(x_1,\dots,x_n | \tau_{o} K_{o}) \big(\pi(\tau|K)\pi(K)\pi(\beta)\pi(\sigma)\big)_{old} d\tau dK}{\int f(x_1, \dots ,x_n | \tau_{o} K_{o}) \big(\pi(\tau|K)\pi(K)\pi(\beta)\pi(\sigma)\big)_{old} d \tau dK d \beta d\sigma }\Big] q(\tau_{n} K_{n}| \tau_{o} K_{o})} 
\end{align*}

Basing priors of the BARS paper (Kass \& Wasserman, 1995)
we have that \\ $\pi(\theta) = \pi(\tau|K)\pi(K)\pi(\beta)\pi(\sigma)$ for both the $\theta_{n}$ and the $\theta_{o}$. \\
$\pi(\beta)$ is an unit information prior, multivariate normal\\
$\pi(\sigma)$ is an inverse gamma  \\
$\pi(\tau | K)$ might be uniform \\
$\pi(K)$ might be a Poisson or uniform 


\subsubsection{BIC}
Applying this ratio into our own model we need to first take the log such that,

\begin{align*}
ratio &= \frac{\Big[ f(x_1,\dots,x_n | \tau_{n} K_{n}) %likelihood new
\big(\pi(\tau|K)\pi(K)\pi(\beta)\pi(\sigma)\big)_{new}%priors for new 
\Big] 
q(\tau_{o} K_{o} | \tau_{n} K_{n})} %proposal density going to old given new 
{\Big[ f(x_1,\dots,x_n | \tau_{o} K_{o}) %likelihood old 
\big(\pi(\tau|K)\pi(K)\pi(\beta)\pi(\sigma)\big)_{old}\Big] %priors for old
q(\tau_{n} K_{n}| \tau_{o} K_{o})} %proposal density going to new given old
\\
log(ratio) &= log \Big[\frac{\Big[ f(x_1,\dots,x_n | \tau_{n} K_{n}) %likelihood new
\big(\pi(\tau|K)\pi(K)\pi(\beta)\pi(\sigma)\big)%priors for new 
\Big] 
q(\tau_{o} K_{o} | \tau_{n} K_{n})} %proposal density going to old given new 
{\Big[ f(x_1,\dots,x_n | \tau_{o} K_{o}) %likelihood old 
\big(\pi(\tau|K)\pi(K)\pi(\beta)\pi(\sigma)\big)\Big] %priors for old
q(\tau_{n} K_{n}| \tau_{o} K_{o})} \Big]%proposal density going to new given old
\\
& =\Big[ log \big[ f(x_1,\dots,x_n | \tau_{n} K_{n})
\big] - 
log \big[ f(x_1,\dots,x_n | \tau_{o} K_{o}) \big] \Big]  \\
& \  \ \ \ \ \ \ \  \ \ \ \ \ \ \ \  +  \Big[ log \big[ q(\tau_{o} K_{o} | \tau_{n} K_{n}) \big] - log \big[ q(\tau_{n} K_{n}| \tau_{o} K_{o})  \big] \Big] 
\end{align*}

From the knowledge gained by the (Kass $\&$ Wasserman, 1995) paper we have that $\Big[ log \big[ f(x_1,\dots,x_n | \tau_{n} K_{n}) \big] -  log \big[ f(x_1,\dots,x_n | \tau_{o} K_{o})\big] \Big] $ approximates BIC. Thus,
\begin{align*}
log(ratio) &= \frac{- \Delta BIC}{2} + 
\Big[ log \big[ q(\tau_{o} K_{o} | \tau_{n} K_{n}) \big] - log \big[ q(\tau_{n} K_{n}| \tau_{o} K_{o})  \big] \Big]
\end{align*}
The $q$ values depend on whether the step is addition or subtraction. \\
\textbf{Addition:} 
$q(\tau_{o} K_{o} | \tau_{n} K_{n}) = c \cdot d \cdot dpois(K_{old} , \lambda)$, $q(\tau_{n} K_{n} | \tau_{o} K_{o}) = c \cdot b \cdot dpois(K_{old} , \lambda)$
\\
\textbf{Subtraction:}
$q(\tau_{o} K_{o} | \tau_{n} K_{n}) = c \cdot b \cdot dpois(K_{new} , \lambda)$, $q(\tau_{n} K_{n} | \tau_{o} K_{o}) = c \cdot d \cdot dpois(K_{old} , \lambda)$ \\


\subsection{Derivations of $\beta$ and $\sigma$ draws}

The posterior for the $\beta$ coefficient is laid out in detail in the \textit{Forecasting time series} (Pesaran Paper, 2006) such that,  $y \sim  N(x \beta , \sigma ) \longrightarrow \beta | \sigma^2, b_0, B_0, V_0, d_0, p , S_{\gamma}, Y_{\gamma} \sim N( \overline{\beta_j } , \overline{V_j} )$
Where $\overline{V}_j = (\sigma^{-2}x^Tx + B_0^{-1})^{-1}$, and $\overline{\beta}_j = \overline{V}_j(\sigma^{-2}x^Ty + B_0^{-1}b_0)$. Thus we have that $\beta  \sim N(\overline{\beta}_j, \overline{V}_j)$. 

The \textit{Forecasting time series} (Pesaran Paper, 2006) also lays out the $\sigma$ posterior. $y \sim N(x \beta , \sigma )$ and $\sigma_j^{-2} \sim  \Gamma(v_0, d_0) \longrightarrow \sigma^{-2}_j | \beta, b_0, B_0, v_0, d_0, p , S_{\gamma}, Y_{\gamma} \sim \Gamma ( \overline{v}_0,  \overline{d}_0) $
Where $\overline{v}_0 = v_0 + \frac{n_j}{2}$, and $\overline{d}_0 = d_0 + \frac{1}{2}(y-x\beta)^T(y-x\beta)$


\section{Methods}

\section{Results}

\section{Discussion}

\section{Appendix}
%%% Acknowledgements (if any)
%%% ------------------------------------------
\section*{Acknowledgements}
We want to thank\ldots


%%% References if bibTeX is used
%%%
%%% Please, do not specify any \bibliographystyle{} command!
%%%
%%% It is already specified in the smj.cls and its
%%% second specification here causes error.
%%% ------------------------------------------------------------
\bibliography{smj-template}


%%% References (if created by hand).
%%% -----------------------------------------------------------------------------------
%\begin{thebibliography}{99}
%\bibitem[Fahrmeir et~al.(2013)]{Fahrmeiret13}
%Fahrmeir, L., Kneib, T., Lang, S., and Marx, B. (2013).
%\textit{Regression: Models, Methods and Applications}.
%Springer-Verlag, New York.
%
%\bibitem[G\'omez et al.(2009)]{Gomezet09}
%G\'omez, G., Luz Calle, M., Oller, R. and Langohr, K. (2009).
%Tutorial on methods for interval-censored data and their implementation in {R}.
%\textit{Statistical Modelling}, \textbf{9}, 259--297.
%
%\bibitem[Kneib(2013)]{Kneib13}
%Kneib, T. (2013). Beyond mean regression. \textit{Statistical Modelling}, \textbf{13}, 275--303.
%
%\bibitem[Kom\'arek and Lesaffre(2006)]{KomarekLesaffre06}
%Kom\'arek, A. and Lesaffre, E. (2006). Bayesian semi-parametric accelerated failure time model for paired doubly-interval-censored data.
%\textit{Statistical Modelling}, \textbf{6}, 3--22.
%
%\bibitem[Lesaffre et~al.(2009)]{Lesaffreet09}
%Lesaffre, E., Kom\'arek, A., and Jara A. (2009)
%The Bayesian approach.
%In Lesaffre, E., Feine, J., Leroux, B., and Declerck, D., eds. 
%\textit{Statistical and Methodological Aspects of Oral  Health Research},
%pages 315--338.
%John Wiley and Sons, Chichester.
%
%\bibitem[Li et~al.(2007)]{Liet07}
%Li, L., Simonoff, J. S., and Tsai, C.-L. (2007)
%Tobit model estimation and sliced inverse regression.
%\textit{Statistical Modelling}, \textbf{7}, 107--123.
%
%\bibitem[Waldmann et~al.(2013)]{Waldmannet13}
%Waldmann, E., Kneib, T., Yue, Y. R., Lang, S., and Flexeder, C. (2013)
%Bayesian semiparametric additive quantile regression.
%\textit{Statistical Modelling}, \textbf{13}, 223--252.
%\end{thebibliography}

\end{document}
