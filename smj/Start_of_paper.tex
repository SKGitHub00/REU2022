%%%
%%%  LaTeX template for publications
%%%  to be submitted to Statistical Modelling
%%%
%%%  Prepared by Arnost Komarek
%%%  Version 0.2 (20140214)
%%%    0.2:  style of references slightly changed,
%%%          support for use with bibTeX added
\documentclass[submit]{smj}


%%%%% PREAMBLE
%%%%% =============================================================================


%%% Place for putting personal \usepackage and \newcommand commands
%%% Note that some packages are loaded automatically
%%% with the smj class. 
%%% These include: graphicx, color, fancyvrb, amsmath, amssymb, calc, upquote (if available), natbib, url, hyperref.
%%%
%%% Please, specify all your personal definitions, newcommand etc. here
%%% and not inside the main body of the text.
%%% -------------------------------------------------------------------------------
%\usepackage{PACKAGE}
%\newcommand{MYCOMMAND}{...}


%%% Identification of authors
%%% -------------------------------------------------------------------------------
%%% For each author, provide his/her first name, surname and possibly initials 
%%% of the middle names. 
%%%
%%% Use \Affil{NUMBER} following the author name for each unique affiliation,
%%% where NUMBER is integer starting from 1 to the number of affiliations needed
%%% in this paper. In case of multiple affiliations of one author, use
%%% \Affil{NUMBER1,}\Affil{NUMBER2,}\Affil{NUMBER3} following the author's name
%%% as it is done for Emmanuel Lesaffre below.

  %%% For papers with 3 or more authors:
  %%%  in \Author{}, separate the authors with commas, the last author is separated by `and' without a comma,
  %%%  in \AuthorRunning{}, use the full name of the first author followed by \textrm{et al.}.
\Author{Kathryn Haglich\Affil{1}, Jeffrey Liebner\Affil{1},
        Sarah Neitzel\Affil{2}, 
        and Amy Pitts\Affil{3}
}
\AuthorRunning{Haglich, Liebner, Neitzel, Pitts}

  %%% For papers with 2 authors:
  %%%  in both \Author{} and \AuthorRunning{},
  %%%  use the full names of both authors separated by 'and' without a comma.
%\Author{Arno\v{s}t Kom\'arek\Affil{1} and Brian Marx\Affil{2}}
%\AuthorRunning{Arno\v{s}t Kom\'arek and Brian Marx}
  
  %%% For papers with 1 author:
  %%%  in both \Author{} and \AuthorRunning{},
  %%%  use the full name the author.
%\Author{Arno\v{s}t Kom\'arek\Affil{1}}
%\AuthorRunning{Arno\v{s}t Kom\'arek}


%%% Affiliations as they should appear on the title page.
%%% -------------------------------------------------------------------------------
%%% Do not provide the full addresses here.
%%% The ordering inside \Affiliations{} should correspond to NUMBERs used 
%%% in \Affil{} commands in \Author{}
\Affiliations{

  %%% 1
\item Department of Mathematics, 
      Lafayette College,
      Easton,
      Pennsylvania, USA

 %%% 2  
\item School of Biodiversity Conservation,
      Unity College, 
      Unity,
      Maine, USA

  %%% 3  
\item Department of Mathematics,
      Marist College,
      Poughkeepsie
      NY, USA

  
%\item Department of Biostatistics,
%      Erasmus University Rotterdam,
%      Rotterdam,
%      the Netherlands

  %%% 5
%\item Interuniversity Institute for Biostatistics and Statistical Bioinformatics,
%      KU Leuven and Universiteit Hasselt,
%      Leuven,
%      Belgium
}   %% end \Affiliations


%%% Postal, e-mail address, phone and fax of the corresponding author (not necessarily the first author).
%%% ------------------------------------------------------------------------------------------------------
%%% Use command \CorrAddress{} to provide a full postal address of the
%%% corresponding author in the form
%%% "Firstname Lastname, Department, University, Street 1, ZIP City, Country" 
%%% Use command \CorrEmail{} to provide an e-mail address of the corresponding author.
%%% Use command \CorrPhone{} to provide a phone number (including the country code!) of the corresponding author.
%%% Use command \CorrFax{} to provide a fax number (including the country code!) of the corresponding author.
\CorrAddress{Jeffrey Liebner, 
             Department of Mathematics, 
             Lafayette College, 
             Easton,
             Pennsylvania, 
        		USA}
\CorrEmail{liebnerj@lafayette.edu}
\CorrPhone{(+420)\;221\;913\;282}
\CorrFax{(+420)\;222\;323\;316}


%%% Title and a short title (to be used as a running header) of the paper
%%% -------------------------------------------------------------------------------
\Title{Developing a Bayesian method for locating breakpoints in time series data.}
\TitleRunning{Breakpoints in time series}


%%% Abstract
%%% -------------------------------------------------------------------------------
\Abstract{
An abstract of up to 200 words should precede the text together with 5 or 6 keywords in alphabetical order 
to describe the content of the paper. Authors should take great care in preparing the abstract and not simply 
lift it from the main text. The abstract should describe the background and contribution of the manuscript 
and give a~clear verbal description of the results and examples, and avoid citations whenever possible. 
Any acknowledgements will be printed at the end of the text.
}


%%% Key words
%%% -------------------------------------------------------------------------------
\Keywords{
AR; Bayesian; Breakpoints; Time Series; 
}


%%%%% MAIN BODY 
%%%%% =============================================================================
\begin{document}


%%% Title page
%%% -------------------------------------------------------------------------------
%%% Use command\maketitle to produce the title page.
\maketitle


%%% Main text
%%% ------------------------------------------
\section{Introduction}

\section{Method}
The Metropolis Hastings algorithm is a mechanism consisting of a Markov Chain Monte Carlo (MCMC) that samples a distribution.  Our MCMC is an adaptation of BARS that have three different overarching step type, death, birth, and move. This process repeatable proposes breakpoint sets which a ratio then determines to accept or not. 

\subsection{Step Type}
The first type of step is we will discuses is death.  This death step takes one existing break point and then deletes it.  The second type of step is Birth. This birth step propose a breakpoint in a random location. There are a couple of constraints on locations of acceptable proposals. A proposed breakpoint can not be an endpoint, an already existing breakpoint, or any data point that is 2 points away from an existing breakpoint. These constraints are necessary in being able to fit an auto-regressive (AR) model. The last type of step is move. Our move step is comprised of two type of moves, jump and jiggle.  Jump has a 25 \% chance of occurring and it is basically a death step and then an birth step. Jiggle has 75\% of occurring and it moves an existing breakpoint in a small defined interval specified by the user. 

\subsection{Probabilities on Steps}
The probabilities of choosing one step over another is specified by the user. However, the probability of doing a death and birth step is always equal to each other.  When a type of step is selected based of user input, then the Metropolis Hastings ratio determinate if the proposed breakpoint set is accepted.  

\subsubsection{BIC} 
The general Metropolis Hastings ratio is the product of the Bayes factor, determined by the ratio of the posteriors, $g$, and the ratio of the Markov Chain Monte Carlo (MCMC) proposal densities, $q$, whose values depend on the current MCMC step. 
\begin{align*}
ratio &= \frac{g(\tau_{n} K_{n} | x_1,\dots,x_t) }{g(\tau_{o} K_{o} | x_1,\dots,x_t)} \times \frac{q(\tau_{o} K_{o} | \tau_{n} K_{n})}{q(\tau_{n} K_{n}| \tau_{o} K_{o})}
\end{align*}
To be able to adequately analysis these ratios we need to put the ratio on a logarithmic scale.  
\begin{align*}
log(ratio) & =\Big[ log \big[ g(\tau_{n} K_{n} | x_1,\dots,x_t)
\big] - log \big[ g(\tau_{o} K_{o} | x_1,\dots,x_t)\big] \Big] \\
& \ \ \ \ \ \ \ \ \ \ \ \ + 
\Big[ log \big[ q(\tau_{o} K_{o} | \tau_{n} K_{n}) \big] - log \big[ q(\tau_{n} K_{n}| \tau_{o} K_{o})  \big] \Big] \\
& \ …
\end{align*}
As shown by Kass and Wasserman (1995), the log of the Bayes Factor can be approximated with BIC with an error on the order of $O(n^{-1/2})$ when the data size is greater than 25 and the prior follows a normal distribution.
Therefore, 
\begin{align*}
 log \big[ g(\tau_{n} K_{n} | x_1,\dots,x_t)
\big] - log \big[ g(\tau_{o} K_{o} | x_1,\dots,x_t)\big]  \approx \frac{- \Delta BIC}{2} 
\end{align*}
which means that 
\begin{align*}
log(ratio) \approx \frac{- \Delta BIC}{2} + 
\Big[ log \big[ q(\tau_{o} K_{o} | \tau_{n} K_{n}) \big] - log \big[ q(\tau_{n} K_{n}| \tau_{o} K_{o})  \big] \Big]
\end{align*}
In the case of birth, 
\begin{align*} 
q(\tau_{o} K_{o} | \tau_{n} K_{n}) = c \cdot d \cdot Poisson(K_{old} , \lambda), \ \ \ q(\tau_{n} K_{n} | \tau_{o} K_{o}) = c \cdot b \cdot Poisson(K_{old} , \lambda).
\end{align*}
When the chosen MCMC step is death,  \begin{align*}
q(\tau_{o} K_{o} | \tau_{n} K_{n}) = c \cdot b \cdot Poisson(K_{new} , \lambda) , \ \ \ q(\tau_{n} K_{n} | \tau_{o} K_{o}) = c \cdot d \cdot Poisson(K_{old} , \lambda).
\end{align*} 
Given the equations above we have that $c$ is the combined probability of doing an addition and subtraction step. $b$ is the balancing birth coefficient and $d$ a balancing death coefficient. They are in place to set the ratio of birth step to death steps. Specifically,
\begin{align*}
b &= \frac{A_{start}}{A_{start} + K_{start} + 1} \times \frac{1}{A}
\end{align*}
\begin{align*}
d &= \frac{K_{start}}{A_{start} + K_{start} + 1} \times \frac{1}{K}
\end{align*}
The first fraction is based off of starting conditions and the second fraction changes through each step. We have that $A_{start}$ is the starting number of available spaces. An available space is any data point that is not itself a breakpoint, and endpoint, or 2 points away from an existing breakpoint.  $K_{start}$ is the starting number of breakpoints that is proposed before the function is called. 


\subsection{AR model and draws}
Once a step has been completed and a new breakpoint set is proposed then the data is fit using an auto-regressive model.  With this information then we can get a draw of the $\beta$ coefficients and $\sigma$.

\subsection{Derivations of $\beta$ and $\sigma$ draws}

The posterior for the $\beta$ coefficient is laid out in detail in the \textit{Forecasting time series} (Pesaran Paper, 2006). 
Given that $b_0$ is the mean of the $\beta$s, 
$B_0$ is the variance co-variance matrix of the $\beta$s for the prior. Also $v_0$ and $d_0$ are the parameters of the inverse gamma prior of the inverse gamma squared (one being the shape the other rate). While $S_t$ is the current state of the break locations, and $y_t$ is the actual data values. 
\begin{align*}
\beta | \sigma^2, b_0, B_0, V_0, d_0 , S_{t}, y_{t} \sim N( \overline{\beta_j } , \overline{V_j} )
\end{align*}
Where 
\begin{align*}
\overline{V}_j = (\sigma^{-2}x^Tx + B_0^{-1})^{-1}, \ \ \  \overline{\beta}_j = \overline{V}_j(\sigma^{-2}x^Ty_t + B_0^{-1}b_0)
\end{align*}
The \textit{Forecasting time series} (Pesaran Paper, 2006) also lays out the $\sigma$ posterior such that 
\begin{align*}
\sigma_j^{-2} \sim  \Gamma(v_0, d_0) \longrightarrow \sigma^{-2}_j | \beta, b_0, B_0, v_0, d_0 , S_{t}, y_{t} \sim \Gamma ( \overline{v}_0,  \overline{d}_0)
\end{align*}
Where 
\begin{align*}
\overline{v}_0 = v_0 + \frac{n_j}{2} , \ \ \  \overline{d}_0 = d_0 + \frac{1}{2}(y_t-x\beta)^T(y_t-x\beta)
\end{align*}


\subsection{Simulation to evaluate}

\section{Results}

\section{Discussion}

\section{Appendix}
%%% Acknowledgements (if any)
%%% ------------------------------------------
\section*{Acknowledgements}
We want to thank\ldots


%%% References if bibTeX is used
%%%
%%% Please, do not specify any \bibliographystyle{} command!
%%%
%%% It is already specified in the smj.cls and its
%%% second specification here causes error.
%%% ------------------------------------------------------------
\bibliography{smj-template}


%%% References (if created by hand).
%%% -----------------------------------------------------------------------------------
%\begin{thebibliography}{99}
%\bibitem[Fahrmeir et~al.(2013)]{Fahrmeiret13}
%Fahrmeir, L., Kneib, T., Lang, S., and Marx, B. (2013).
%\textit{Regression: Models, Methods and Applications}.
%Springer-Verlag, New York.
%
%\bibitem[G\'omez et al.(2009)]{Gomezet09}
%G\'omez, G., Luz Calle, M., Oller, R. and Langohr, K. (2009).
%Tutorial on methods for interval-censored data and their implementation in {R}.
%\textit{Statistical Modelling}, \textbf{9}, 259--297.
%
%\bibitem[Kneib(2013)]{Kneib13}
%Kneib, T. (2013). Beyond mean regression. \textit{Statistical Modelling}, \textbf{13}, 275--303.
%
%\bibitem[Kom\'arek and Lesaffre(2006)]{KomarekLesaffre06}
%Kom\'arek, A. and Lesaffre, E. (2006). Bayesian semi-parametric accelerated failure time model for paired doubly-interval-censored data.
%\textit{Statistical Modelling}, \textbf{6}, 3--22.
%
%\bibitem[Lesaffre et~al.(2009)]{Lesaffreet09}
%Lesaffre, E., Kom\'arek, A., and Jara A. (2009)
%The Bayesian approach.
%In Lesaffre, E., Feine, J., Leroux, B., and Declerck, D., eds. 
%\textit{Statistical and Methodological Aspects of Oral  Health Research},
%pages 315--338.
%John Wiley and Sons, Chichester.
%
%\bibitem[Li et~al.(2007)]{Liet07}
%Li, L., Simonoff, J. S., and Tsai, C.-L. (2007)
%Tobit model estimation and sliced inverse regression.
%\textit{Statistical Modelling}, \textbf{7}, 107--123.
%
%\bibitem[Waldmann et~al.(2013)]{Waldmannet13}
%Waldmann, E., Kneib, T., Yue, Y. R., Lang, S., and Flexeder, C. (2013)
%Bayesian semiparametric additive quantile regression.
%\textit{Statistical Modelling}, \textbf{13}, 223--252.
%\end{thebibliography}

\end{document}
