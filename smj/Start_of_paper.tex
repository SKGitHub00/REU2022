%%%
%%%  LaTeX template for publications
%%%  to be submitted to Statistical Modelling
%%%
%%%  Prepared by Arnost Komarek
%%%  Version 0.2 (20140214)
%%%    0.2:  style of references slightly changed,
%%%          support for use with bibTeX added
\documentclass[submit]{smj}


%%%%% PREAMBLE
%%%%% =============================================================================


%%% Place for putting personal \usepackage and \newcommand commands
%%% Note that some packages are loaded automatically
%%% with the smj class. 
%%% These include: graphicx, color, fancyvrb, amsmath, amssymb, calc, upquote (if available), natbib, url, hyperref.
%%%
%%% Please, specify all your personal definitions, newcommand etc. here
%%% and not inside the main body of the text.
%%% -------------------------------------------------------------------------------
%\usepackage{PACKAGE}
%\newcommand{MYCOMMAND}{...}


%%% Identification of authors
%%% -------------------------------------------------------------------------------
%%% For each author, provide his/her first name, surname and possibly initials 
%%% of the middle names. 
%%%
%%% Use \Affil{NUMBER} following the author name for each unique affiliation,
%%% where NUMBER is integer starting from 1 to the number of affiliations needed
%%% in this paper. In case of multiple affiliations of one author, use
%%% \Affil{NUMBER1,}\Affil{NUMBER2,}\Affil{NUMBER3} following the author's name
%%% as it is done for Emmanuel Lesaffre below.

  %%% For papers with 3 or more authors:
  %%%  in \Author{}, separate the authors with commas, the last author is separated by `and' without a comma,
  %%%  in \AuthorRunning{}, use the full name of the first author followed by \textrm{et al.}.
\Author{Kathryn Haglich\Affil{1}, Jeffrey Liebner\Affil{1},
        Sarah Neitzel\Affil{2}, 
        and Amy Pitts\Affil{3}
}
\AuthorRunning{Haglich, Liebner, Neitzel, Pitts}

  %%% For papers with 2 authors:
  %%%  in both \Author{} and \AuthorRunning{},
  %%%  use the full names of both authors separated by 'and' without a comma.
%\Author{Arno\v{s}t Kom\'arek\Affil{1} and Brian Marx\Affil{2}}
%\AuthorRunning{Arno\v{s}t Kom\'arek and Brian Marx}
  
  %%% For papers with 1 author:
  %%%  in both \Author{} and \AuthorRunning{},
  %%%  use the full name the author.
%\Author{Arno\v{s}t Kom\'arek\Affil{1}}
%\AuthorRunning{Arno\v{s}t Kom\'arek}


%%% Affiliations as they should appear on the title page.
%%% -------------------------------------------------------------------------------
%%% Do not provide the full addresses here.
%%% The ordering inside \Affiliations{} should correspond to NUMBERs used 
%%% in \Affil{} commands in \Author{}
\Affiliations{

  %%% 1
\item Department of Mathematics, 
      Lafayette College,
      Easton,
      Pennsylvania, USA

 %%% 2  
\item School of Biodiversity Conservation,
      Unity College, 
      Unity,
      Maine, USA

  %%% 3  
\item Department of Mathematics,
      Marist College,
      Poughkeepsie
      New York, USA

  

}   %% end \Affiliations


%%% Postal, e-mail address, phone and fax of the corresponding author (not necessarily the first author).
%%% ------------------------------------------------------------------------------------------------------
%%% Use command \CorrAddress{} to provide a full postal address of the
%%% corresponding author in the form
%%% "Firstname Lastname, Department, University, Street 1, ZIP City, Country" 
%%% Use command \CorrEmail{} to provide an e-mail address of the corresponding author.
%%% Use command \CorrPhone{} to provide a phone number (including the country code!) of the corresponding author.
%%% Use command \CorrFax{} to provide a fax number (including the country code!) of the corresponding author.
\CorrAddress{Jeffrey Liebner, 
             Department of Mathematics, 
             Lafayette College, 
             Easton,
             Pennsylvania, 
        		USA}
\CorrEmail{liebnerj@lafayette.edu}
\CorrPhone{(+420)\;221\;913\;282}
\CorrFax{(+420)\;222\;323\;316}


%%% Title and a short title (to be used as a running header) of the paper
%%% -------------------------------------------------------------------------------
\Title{Developing a Bayesian method for locating breakpoints in time series data.}
\TitleRunning{Breakpoints in time series}


%%% Abstract
%%% -------------------------------------------------------------------------------
\Abstract{
An abstract of up to 200 words should precede the text together with 5 or 6 keywords in alphabetical order 
to describe the content of the paper. Authors should take great care in preparing the abstract and not simply 
lift it from the main text. The abstract should describe the background and contribution of the manuscript 
and give a~clear verbal description of the results and examples, and avoid citations whenever possible. 
Any acknowledgements will be printed at the end of the text.
}


%%% Key words
%%% -------------------------------------------------------------------------------
\Keywords{
Breakpoints; Time Series; Bayesian; AR; BARS
}


%%%%% MAIN BODY 
%%%%% =============================================================================
\begin{document}


%%% Title page
%%% -------------------------------------------------------------------------------
%%% Use command\maketitle to produce the title page.
\maketitle


%%% Main text
%%% ------------------------------------------
\section{Introduction}

\section{Method}
The Metropolis Hastings algorithm is a mechanism consisting of a Markov Chain Monte Carlo (MCMC) that samples a distribution.  Our MCMC is an adaptation of BARS that have three different overarching step types: birth, death and move. This process repeatable proposes breakpoint sets which a ratio then determines whether or not it should be accepted. 

\subsection{Step Type}

The birth step randomly proposes a breakpoint at an available location. An available location is where a breakpoint could be placed given the following constraints. First, the location cannot have a breakpoint or an endpoint currently assigned to it. Second, for linear fits and AR(1) models, the location must be at least two data points away from the breakpoints closest to the particular location. For AR(p) models, the minimum distance away a location must be from its closest breakpoints is $2p$. If a location is in accordance with these constrains, then it is an available location.  

The death step randomly chooses an existing breakpoint and proposes a set without that chosen breakpoint. 

The general move step is a subtraction step followed immediately by an addition step and can be broken down to two specific types of move: jump and jiggle. Jump allows the movement of a breakpoint to any available location. Jiggle restricts the distance a breakpoint can move to a jiggle neighborhood, an interval surrounding the breakpoint's original location. To calculate the jiggle neighborhood, $J_n$, 
\begin{align*}
J_n = ( x_b-pn, x_b+pn )
\end{align*}
where $x_b$ is the original location of the chosen breakpoint, $n$ is the size of the data set, and $p$ is the user-inputed percent in decimal form. When a move step is chosen, there is a $\zeta$ probability that a jiggle will be performed, which is determined by the user such that $0<\zeta<1$ and $\zeta \in \mathbb{Q}$. The probability of a jump occurring is $1-\zeta$. 

\subsection{Probabilities of the Steps}
The combined probabilities of performing a birth step, $b_p$, and a death step, $d_p$, is equal to the user imputed value, $c$ such that $c \in \mathbb{Q}$ and $0 < c < 1$. The ratio of birth steps to death steps is determined by $c$ and the initial conditions of the starting number of breakpoints, $K_{start}$, and the starting number of available spaces, $A_{start}$. From this, the following equations can be derived for $b_p$ and $d_p$: 
\begin{align*}
b_p = c  \ \frac{A_{start}}{A_{start}+ K_{start}+1}  \ \ \ \ d_p = c \  \frac{K_{start}+1}{A_{start}+ K_{start}+1}
\end{align*}
Then, the probability of a specific birth step given $A$ available locations, $b$, is the equation 
\begin{align*}
b &= \ b_p \times \frac{1}{A}
\end{align*}
Thus, the probability of a specific death step given $K$ breakpoints, $d$, is the equation
\begin{align*}
d &= \ d_p \times \frac{1}{K}
\end{align*}
The probability of a move step, $m$ is represented by the equation $m = 1-c$. The probability of jiggle, $jg$, and the probability of jump, $ju$, are calculated by the following equations: 
\begin{align*}
jg = m\zeta \ \ \ \ \ ju = 1-jg
\end{align*}

\subsubsection{Metropolis Hastings Ratio and BIC Approximation} 
After a specific step is selected, the Metropolis Hastings ratio, as derived below, is used to determine the acceptance of the proposed breakpoint set. To determine the thresh hold of acceptance,  $r_{unif}$ is generated from a uniform distribution from a sample space on the interval (0,1). If the ratio is greater than $r_{unif}$, then the proposed breakpoint set is accepted and kept. Otherwise, the old breakpoint set is retained. 

The general Metropolis Hastings ratio is the product of the Bayes factor, determined by the ratio of the posteriors, $g$, and the ratio of the Markov Chain Monte Carlo (MCMC) proposal densities, $q$, whose values depend on the current MCMC step. 
\begin{align*}
ratio &= \frac{g(\tau_{n} K_{n} | x_1,\dots,x_t) }{g(\tau_{o} K_{o} | x_1,\dots,x_t)} \times \frac{q(\tau_{o} K_{o} | \tau_{n} K_{n})}{q(\tau_{n} K_{n}| \tau_{o} K_{o})}
\end{align*}
When the log likelihood of the equation is taken, 
\begin{align*}
log(ratio) & =\Big[ log \big[ g(\tau_{n} K_{n} | x_1,\dots,x_t)
\big] - log \big[ g(\tau_{o} K_{o} | x_1,\dots,x_t)\big] \Big] \\
& \ \ \ \ \ \ \ \ \ \ \ \ + 
\Big[ log \big[ q(\tau_{o} K_{o} | \tau_{n} K_{n}) \big] - log \big[ q(\tau_{n} K_{n}| \tau_{o} K_{o})  \big] \Big] \\
& \ …
\end{align*}
As shown by Kass and Wasserman (1995), the log of the Bayes Factor can be approximated with BIC with an error on the order of $O(n^{-1/2})$ when the data size is greater than 25 and the prior follows a normal distribution.
Therefore, 
\begin{align*}
 log \big[ g(\tau_{n} K_{n} | x_1,\dots,x_t)
\big] - log \big[ g(\tau_{o} K_{o} | x_1,\dots,x_t)\big]  \approx \frac{- \Delta BIC}{2} 
\end{align*}
which means that 
\begin{align*}
log(ratio) \approx \frac{- \Delta BIC}{2} + 
\Big[ log \big[ q(\tau_{o} K_{o} | \tau_{n} K_{n}) \big] - log \big[ q(\tau_{n} K_{n}| \tau_{o} K_{o})  \big] \Big]
\end{align*}
In the case of a birth step, 
\begin{align*} 
q(\tau_{o} K_{o} | \tau_{n} K_{n}) = c \cdot d \cdot Poisson(K_{old} , \lambda), \ \ \ q(\tau_{n} K_{n} | \tau_{o} K_{o}) = c \cdot b \cdot Poisson(K_{old} , \lambda).
\end{align*}
In the case of a death step,  
\begin{align*}
q(\tau_{o} K_{o} | \tau_{n} K_{n}) = c \cdot b \cdot Poisson(K_{new} , \lambda) , \ \ \ q(\tau_{n} K_{n} | \tau_{o} K_{o}) = c \cdot d \cdot Poisson(K_{old} , \lambda).
\end{align*} 
In the case of a move step, irrelevant of whether it is specifically jiggle or jump, 
\begin{align*} 
log \big[ q(\tau_{o} K_{o} | \tau_{n} K_{n}) \big] - log \big[ q(\tau_{n} K_{n}| \tau_{o} K_{o})  \big] = 0 
\end{align*}
Henceforth, for a move, 
\begin{align*}
log(ratio) \approx \frac{- \Delta BIC}{2} 
\end{align*}

\subsection{AR model and draws}
Once a step has been completed and a new breakpoint set is proposed then the data is fit using an auto-regressive model.  With this information then we can get a draw of the $\beta$ coefficients and $\sigma$.

\subsection{Derivations of $\beta$ and $\sigma$ draws}

As determined by Peseran (2006), the posterior for the $\beta$ coefficients is 
\begin{align*}
\beta | \sigma^2, b_0, B_0, v_0, d_0 , S_{t}, y_{t} \sim N( \overline{\beta_j } , \overline{V_j} )
\end{align*}
where 
\begin{align*}
\overline{V}_j = (\sigma^{-2}x^Tx + B_0^{-1})^{-1}, \ \ \  \overline{\beta}_j = \overline{V}_j(\sigma^{-2}x^Ty_t + B_0^{-1}b_0).
\end{align*}
The conditions are the following: $b_0$ is the mean of the $\beta$ coefficients, $B_0$ is the variance co-variance matrix of the $\beta$ coefficients for the prior, $v_0$ and $d_0$ are the parameters of the inverse gamma prior of the inverse gamma squared (one being the shape the other rate), $S_t$ is the current breakpoint set, and $y_t$ is the actual data values. 
Pesaran (2006) derives the $\sigma$ posterior such that 
\begin{align*}
\sigma_j^{-2} \sim  \Gamma(v_0, d_0) \longrightarrow \sigma^{-2}_j | \beta, b_0, B_0, v_0, d_0 , S_{t}, y_{t} \sim \Gamma ( \overline{v}_0,  \overline{d}_0)
\end{align*}
where 
\begin{align*}
\overline{v}_0 = v_0 + \frac{n_j}{2} , \ \ \  \overline{d}_0 = d_0 + \frac{1}{2}(y_t-x\beta)^T(y_t-x\beta).
\end{align*}


\subsection{Simulations to evaluate}

\section{Results}

\section{Discussion}

\section{Appendix}
%%% Acknowledgements (if any)
%%% ------------------------------------------
\section*{Acknowledgements}
We want to thank\ldots \textbf{JEFF}


%%% References if bibTeX is used
%%%
%%% Please, do not specify any \bibliographystyle{} command!
%%%
%%% It is already specified in the smj.cls and its
%%% second specification here causes error.
%%% ------------------------------------------------------------
\bibliography{smj-template}

\begin{thebibliography}{99}
\bibitem[Bai, J. and Perron, P.,(1998)]{Bai-Perron98}
Bai, J. and Perron, P., (1998).
\textit{Estimating and testing linear models with multiple structural changes}.
Econometrica, pp.47-78.

\bibitem[Bai, J. and Perron, P., (2003)]{Bai-Perron03}
Bai, J. and Perron, P., (2003).
\textit{ Computation and analysis of multiple structural change models}.
Journal of applied econometrics, 18(1), pp.1-22.

\bibitem[Denison, et~al.(2003)]{Denison98}
Denison, D.G.T., Mallick, B.K. and Smith, A.F.M., (1998). 
\textit{Automatic Bayesian curve fitting}. 
Journal of the Royal Statistical Society: Series B (Statistical Methodology), 60(2), pp.333-350.

\bibitem[DiMatteo, et~al..(2001)]{DiMatteo01}
DiMatteo, I., Genovese, C.R. and Kass, R.E., 2001. 
\textit{Bayesian curve‐fitting with free‐knot splines}. 
Biometrika, 88(4), pp.1055-1071.

\bibitem[Gamber, et~al.(2016)]{Gamber16}
Gamber, E.N., Liebner, J.P. and Smith, J.K., (2016). 
I\textit{nflation persistence: revisited}. 
International Journal of Monetary Economics and Finance, 9(1), pp.25-44.

\bibitem[Kass, R.E., Wasserman, L. (1995)]{Kass95}
Kass, R.E. and Wasserman, L., (1995). 
\textit{A reference Bayesian test for nested hypotheses and its relationship to the Schwarz criterion}. 
Journal of the american statistical association, 90(431), pp.928-934.

\bibitem[McLeod, A., Zhang, Y,.(2008)]{McLeod08}
McLeod, A.I. and Zhang, Y., (2008).
\textit{ Improved subset autoregression: With R package}. 
 Journal of Statistical Software, 28(2), pp.1-28.

\bibitem[Pesaran, et~al.(2006)]{Pesaran06}
Pesaran, M.H., Pettenuzzo, D. and Timmermann, A., (2006). 
\textit{Forecasting time series subject to multiple structural breaks}. 
The Review of Economic Studies, 73(4), pp.1057-1084.

\bibitem[R Core Team(2017)]{R17}
R Core Team( 2017). 
\textit{R: A Language and Environment for Statistical Computing}.
R Foundation for Statistical Computing

\bibitem[Ruggieri, E. (2013)]{Ruggieri13}
Ruggieri, E.,( 2013). 
\textit{A Bayesian approach to detecting change points in climatic records}.
International Journal of Climatology, 33(2), pp.520-528.

\bibitem[Wallstrom, et~al.(2008)]{Wallstron08}
Wallstrom, G., Liebner, J. and Kass, R.E., (2008). 
\textit{An implementation of Bayesian adaptive regression splines (BARS) in C with S and R wrappers}. 
Journal of Statistical Software, 26(1), p.1.

\bibitem[Zeileis, et~al. (2007)]{Zeileis07}
Zeileis, A., Leisch, F., Hansen, B., Hornik, K., Kleiber, C. and Zeileis, M.A., (2007). 
\textit{The strucchange Package}.
R manual.

\bibitem[Zhou, S., Shen, X., (2001)]{Zhou01}
Zhou, S. and Shen, X., (2001). 
\textit{Spatially adaptive regression splines and accurate knot selection schemes}.
Journal of the American Statistical Association, 96(453), pp.247-259.

\end{thebibliography}
\end{document}
